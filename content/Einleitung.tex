Evolution\"ar gesehen, ist es die gr\"o{\ss}te St\"arke des Menschen, sich seiner Umgebung oder auch seine Umgebung an sich anzupassen. Daher liegt es in der Natur des Menschen, sich Werkzeuge herzustellen und Erfindungen zu machen, die im Alltag hilfreich sind.

Eine der wichtigsten Grundlagen daf\"ur, denn sie erm\"oglicht ein hohes, abstraktes Verst\"andnis und die F\"ahigkeit, zu lernen, ist die multisensorische Integration, ein evolution\"ares Wunder, das mehrmals innerhalb einer Sekunde geschieht: Unser Gehirn f\"uhrt unbemerkt und scheinbar m\"uhelos die Informationen verschiedener Sinne zu einem f\"ur uns konsistenten Gesamtbild zusammen.
Dass diese Verarbeitung \"uberhaupt stattfindet, bemerken wir eigentlich nur, sobald es dabei zu Komplikationen, das hei{\ss}t, Abweichungen der bisherigen Erfahrung, kommt. 

Eine relativ neue Erfindung ist Virtual Reality, mit der wir erstmalig die Chance haben, unsere Umgebung vollst\"andig nach unseren Vorstellung zu formen, beispielsweise auch mit ver\"anderten physikalischen Gesetzen.
Virtual Reality hat das enorme Potential viele Bereiche der Gesellschaft nachhaltig zu ver\"andern, nur trifft es sich leider, dass genau bei der Integration der Sensorik Schwierigkeiten auftreten, die als Cyber Sickness bekannt sind.