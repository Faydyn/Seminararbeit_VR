Die multisensorische Integration ist ein evolution\"ares Wunder, das viele Male innerhalb einer Sekunde geschieht: 
Unser Gehirn f\"uhrt unbemerkt und scheinbar m\"uhelos die Informationen verschiedener Sinne zu einem f\"ur uns konsistenten Gesamtbild zusammen. 
Dass diese Verarbeitung \"uberhaupt stattfindet, bemerken wir eigentlich nur, sobald es dabei zu Komplikationen kommt. Beispielsweise bei Auto- oder Karusselfahrten, gerade als Kind, wird uns oft schwindelig und \"ubel - daraus l\"asst sich zweierlei ableiten: 
Zum einen handelt es sich hier um Formen der \textit{Bewegung} im Raum, zum anderen scheinbar um einen \textit{Lernprozess}, da eben genannte Probleme oft mit dem Alter abnehmen.\\
F\"ur die Wahrnehmung der Bewegung sind, neben spezieller Sensorik in den Muskeln, vor allem der visuelle und vestibuläre Sinn veranwortlich. 
