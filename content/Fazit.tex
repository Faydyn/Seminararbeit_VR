Cyber Sickness entsteht durch eine Form des visuell-vestibul\"aren Konflikts und ist, auf Grund der entstehenden Symptome, ein zentraler Aspekt der Nutzung von Virtual Reality. Diese k\"onnen das Erlebnis in einer virtuellen Realit\"at unertr\"aglich machen.

Es wurden in \autoref{Maßnahmen gegen CS} eine Reihe verschiedener Methoden vorgestellt, mit deren Hilfe man, vor allem durch Kombination der Methoden und unter Beachtung bestimmter Faktoren, Cyber Sickness reduzieren kann. Das generelle Prinzip der Nat\"urlichkeit oder Vertrautheit, das es einzuhalten gilt, zieht sich durch die Ma{\ss}nahmen.
Dieses Wissen kann helfen, in Zukunft einen besseren, f\"ur den Anwender angenehmeren Umgang mit weniger Cyber Sickness-Symptomen in Virtual Reality zu erm\"oglichen. Dennoch haben diese Methoden nur bedingte G\"ultigkeit und \"Ubertragbarkeit, da es sich hierbei noch eher um Grundlagenforschung handelt.

Jedoch ist es wichtig zu verstehen, dass momentan keine Theorie existiert, die die Ph\"anomene von Cyber Sickness vollst\"andig erkl\"aren kann. Daher ist es auch nicht m\"oglich zu sagen, ob Cyber Sickness und klassische Motion Sickness denselben Ursprung haben. Dennoch beziehen sich viele Studien \"uber Cyber Sickness in Virtual Reality auf \"altere Studien, in denen es eigentlich um klassische Motion Sickness oder Simulator Sickness geht und versuchen \"ahnliche Ergebnisse im Sinne der Sensory Conflict Theory zu finden. Dies hat oft widerspr\"uchliche Ergebnisse zur Folge.


Weiterhin folgt aus dieser "`Vererbung"' von der klassischen Motion Sickness an die Cyber Sickness, dass in der Literatur keine eindeutige Nomenklatur herrscht, in der Begriffe wie Cyber Sickness, Virtual Reality Sickness, Simulator Sickness und Motion Sickness teils synonym verwendet werden, ohne dass dies gerechtfertig ist, da es keine genauen Definitionen f\"ur die jeweiligen Begriffe gibt.

Ausgehend von einer exakteren Benennung wird dann zus\"atzliche Forschung ben\"otigt, um eine bessere Theorie zur Erkl\"arung von Cyber Sickness zu finden. Auch m\"ussen neue Messverfahren erschlossen werden, da viele der Messung aktuell auf Selbstausk\"unften beruhen, welche subjektiv verf\"alscht sein k\"onnen. Durch Umsetzen dieser drei Punkte w\"urden danach eindeutigere, weniger widerspr\"uchliche und vor allem besser vergleichbare Ergebnisse entstehen. Es ist wichtig, das nicht zu vernachl\"assigen, w\"ahrend in der Zwischenzeit mit den aktuellen Gegebenheiten weitergeforscht wird.

Zuletzt sollte bei zuk\"unftiger Forschung und der Umsetzung neuer Methoden an verschiedenste Gruppen gedacht werden, damit keine Ungleichheit zwischen Gruppen herrscht, wie das bei der Passform der Head-Mounted Displays in \autoref{abb:hmdfit} der Fall war. Der Mensch hat sich in seiner Evolution \"uber die Zeit schon an verschiedenste Gegebenheiten angepasst, somit ist anzunehmen, dass sich selbiges auch auf Virtual Reality gilt. Bis dahin m\"ussen wir uns aber selbst bestm\"oglich, durch eben genannte Verbesserung und mit Methoden gegen Cyber Sickness, unterst\"utzen, um das gro{\ss}e Potential, welches Virtual Reality innewohnt, effektiv nutzen zu k\"onnen.

