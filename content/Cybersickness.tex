Das gravierendste Problem, welches bei der Nutzung von Virtual Reality auftreten kann, sind die Symptome der Cyber Sickness. Diese ähneln denen der klassischen Motion Sickness und umfassen eine Vielzahl unangenehmer Empfindungen und Reaktionen durch den betroffenen Organismus: Kopfschmerzen, Schweißausbrüche, Orientierungslosigkeit, Schwindelanfälle, Ataxia und Übelkeit bis hin zum Erbrechen\cite{LaViola:2000:CSinVR, Kolasinski:1998:SympCS}.\\
Durch die Ähnlichkeit in den körperlichen Reaktionen zur klassischen Motion Sickness, die beispielsweise von Auto- oder Schifffahrten bekannt ist, versucht man auch, denselben Erklärungsansatz zu verwenden: die \textit{Sensory Conflict Theory}\cite{Kolasinski:1998:SympCS,Johnson:2005:SCT_Expl}.
Diese postuliert, dass die Symptome auftreten, wenn bei der multimodalen, sensorischen Integration bezüglich der Selbstbewegung inkongruente Reize wahrgenommen wurden und vor allem dann, wenn das aktuelle Empfinden im Widerspruch mit vorherigen Lernerfahrung in ähnlichen Situationen steht\cite{Reason:1975:MSexp}.\\
Für die Wahrnehmung von Bewegung ist die Propriozeption, vor allem aber der Gleichgewichtssinn und Sehsinn zuständig.
Bei Motion Sickness besteht das Problem darin, dass keine passenden visuellen Reize vorhanden sind, wie auf der Innenkabine eines Schiffes bei starkem Wellengang, was bekanntermaßen zu Seekrankheit, eine Form der Motion Sickness, führt.
Im Gegensatz dazu entsteht bei Virtual Reality \textit{Vection}, eine Illusion in der Perzeption der Eigenbewegung, die allein durch visuelle Stimuli entsteht, ohne vestibuläre Stimuli.\\
Zwar liegt beiden, Motion und Cyber Sickness, der \textit{visuell-vestibuläre Konflikt} zu Grunde, jedoch ist die Art, wie dieser entsteht, ebenso wie einige Symptome der beiden, unterschiedlich \cite{Stanney:1997:MSCSSS}. Deswegen wird die Sensory Conflict Theory auch durchaus als Erklärung für die Symptome der Cyber Sickness angezweifelt  \cite{Kolasinski:1998:SympCS}.\\
Eine Alternative stellt die \textit{Theory of Postural Instability} dar: