Virtual Reality erfreut sich immer gr\"o{\ss}erer Beliebtheit. Ein  Nebeneffekt der erw\"unschten Vection, der bei der Nutzung von Virtual Reality h\"aufig auftritt, ist die Cyber Sickness.
Diese entsteht durch eine Inkongruenz bei der sensorischen Integration von visuellen und vestibul\"aren Reizen.
Mediierende Interindividuelle Faktoren sind Alter, Geschlecht, empfundene Kontrolle und Haltungsstabilit\"at.
Die Graphik sollte \"uber hohe Aufl\"osung und Wiederholungsrate sowie niederige Latenz verf\"ugen. Au{\ss}erdem sollte sie statische Element aufweisen. Zudem sollten kongruente vestibul\"are Stimuli erzeugt werden durch VRN-Chairs oder Treadmills. Man kann durch galvanische Vestibul\"arstimulation f\"ur vestibul\"aren Reize ein "`Downweighing"' zu erzeugen, was kurzzeitig die Cyber Sickness reduziert. Dies wird auch durch individualisierbares Equipment erleichtert. Diese Methoden sollten kombiniert angewandt werden.




% In diesem Paper wird darauf eingegangen, wie das Zusammenspiel der Sinne funktioniert und warum dies in Virtual Reality zu Cyber Sickness f\"uhren kann. Es wird dargestellt, welche Vorkehrungen zu ergreifen sind, um dem vorzubeugen, welche Methoden es gibt, um Cyber Sickness zu reduzieren und wie hilfreich sie sind. Schlussendlich wird die aktuelle Situation in der Forschung zu dieser Thematik zusammengefasst und ein Ausblick mit Verbesserungsvorschl\"agen gegeben.