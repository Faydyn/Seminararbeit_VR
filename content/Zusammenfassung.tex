Durch die Digitalisierung der Gesellschaft, sinkende Kosten und gleichzeitig steigende Leistungsf\"ahigkeit der erfordlichen Hardware erfreut sich Virtual Reality immer gr\"oßerer Beliebtheit. Ein ungew\"unschter Nebeneffekt, der bei der Nutzung von Virtual Reality h\"aufig auftritt, ist die Cyber Sickness. In diesem Paper wird darauf eingegangen, wie das Zusammenspiel der Sinne funktioniert und warum das in Virtual Reality zu Cyber Sickness f\"uhren kann. Es wird dargestellt, welche Vorkehrungen zu ergreifen sind, um dem vorzubeugen und welche Methoden es gibt, um Cyber Sickness zu reduzieren und wie hilfreich sie sind. Schlussendlich wird die aktuelle Situation in der Forschung zu dieser Thematik zusammengefasst und ein Ausblick mit Verbesserungsvorschl\"agen gegeben.